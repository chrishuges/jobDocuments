\documentclass[11pt]{article}
\usepackage{array, xcolor, lipsum, bibentry, parallel}
\usepackage[margin=3cm]{geometry}
\usepackage{setspace}
\usepackage{tabularx}
\usepackage{paralist}
\usepackage{scrextend}
\usepackage[colorlinks = true, linkcolor = blue, urlcolor  = blue, citecolor = blue, anchorcolor = blue]{hyperref}


\title{\bfseries\Huge Christopher Hughes}
\author{}
\date{}

\definecolor{lightgray}{gray}{0.8}
\newcolumntype{L}{>{\raggedleft}p{0.14\textwidth}}
\newcolumntype{R}{p{0.8\textwidth}}
\newcommand\VRule{\color{lightgray}\vrule width 0.5pt}


\begin{document}

	\newpage
	\setcounter{page}{1}
	\noindent

\begin{LARGE}

	\noindent\textbf{Christopher Hughes}\vspace{1pt}

\end{LARGE}


\noindent\rule{\textwidth}{1pt}\vspace{9pt}

\noindent
\begin{minipage}{0.5\textwidth}
	\begin{flushleft}
		chughes@bcgsc.ca\\
		Nationality: Canadian\\ \href{https://scholar.google.com/citations?user=jPSwBGwAAAAJ}{Google Scholar Citations}
	\end{flushleft}
\end{minipage}%
\begin{minipage}{0.5\textwidth}
	\begin{flushright}
		2477 Carolina Street, Unit 303\\
		V5T 0G8 Vancouver, Canada\\ \href{https://www.linkedin.com/in/christopher-hughes-612460133/}{LinkedIn Profile}
	\end{flushright}
\end{minipage}\vspace{6pt}


\section*{Relevant Skills}
\begin{addmargin}[3em]{2em}% 1em left, 2em right
	\subsection*{Research and Leadership Abilities}
	\begin{itemize}
		\item Experienced in the initial setup, administration, day-to-day management, as well as the communication, training, and outreach aspects of a mass spectrometry core facility catering to a wide range of individuals.
		\item Highly skilled in the application of DNA/RNA sequencing, proteomics, and metabolomics technologies towards the study of complex biological systems in basic and clinical research settings.
		\item Extensive experience using software tools such as R, Python, and Adobe Illustrator for data analysis and visalization.
		\item Considerable knowledge in the conception and transition of project ideas and preliminary findings into applications for granting agencies, navigating both academic and government policies.
		\item Skilled in the development, management, and execution of small and large scale research projects comprising individuals or collaborative teams of researchers working towards completion of set aims.

		%\item Extensive experience working with a range mass spectrometry
		%architectures and associated hardwares (e.g. LC, UV) and softwares from
		%manufacturers such as Thermo, AB Sciex, Waters, and Agilent.

		%\item Strong leadership skills developed through management of numerous
		%undergraduate, graduate student, and technician research projects.
		%  \item Experienced working with proteomics in clinical settings using patient
		%  derived material, and with the design and execution of large-cohort studies.
		%  \item Significant exposure and experience with proteomics experiment design,
		%  execution, and data analysis, including shotgun, targeted (SRM),
		%  phosphorylation, and intact protein analyses.
		%  \item Extensive experience handling a broad range of mass spectrometer
		%  architectures and platforms, as well as accompanying chromatography systems.
		%  \item Highly skilled in the development of technologies and bioinformatic
		%  solutions for challenging proteomics applications, such as ultrasensitive
		%  analysis of quantity-limited samples and tissues.
		%  \item Experienced in the design, preparation, and bioinformatic analysis of
		%  samples for next-generation sequencing analysis, and their incoporation with
		%  proteomics.
	\end{itemize}
\end{addmargin}

\section*{Current Role}
\begin{tabular}{L!{\VRule}R}
	2018-present&\textbf{Staff Scientist and Mass Spectrometry Service Specialist}\\
	&British Columbia Cancer Research Centre\\
\end{tabular}\\

\begin{addmargin}[7.5em]{2em}% 1em left, 2em right
	In this role with Dr. Poul Sorensen, my responsibilities include:
	\begin{itemize}
		\item Independent conception, development, grant acquisition, execution, and supervision of studies examining the dynamics of protein translation at a sub-cellular level in cancer, with specific sub-studies focused on elucidating the mechanistic roles of targets using combinations of molecular biology, sequencing, proteomics, metabolomics, microscopy, and bioinformatic techniques.
		\item Participation in pan-Canada and global research projects involving the Sorensen lab, including experimental conception and design, data acquisition and analysis, grant acquisition, and student supervision.
	\end{itemize}
\end{addmargin}

\section*{Selected Publications}
{\setlength{\extrarowheight}{4pt}%
\begin{tabular}{L!{\VRule}R}
	2019&\textbf{Hughes, C.S.}, Sorensen, P.H., Morin, G.B. A Standardized and Reproducible Proteomics Protocol for Bottom-up Quantitative Analysis of Protein Samples using SP3 and Mass Spectrometry \textit{Methods in Mol. Biol.}. PMID: 30852816\\
	2019&\textbf{Hughes, C.S.}, Moggridge, S., Mueller, T., Sorensen, P.H., Morin, G.B., Krijgsveld, J. Single-pot, Solid-phase-enhanced Sample Preparation for Proteomics Experiments \textit{Nature Protocols}. PMID: 30464214\\
	2019&Kovalchik, K.A., Colborne, S., Spencer, S., Sorensen, P.H., Chen, D.D.Y., Morin, G.B., \textbf{Hughes, C.S.\textsuperscript{$\diamond$}}, RawTools: Rapid and Dynamic Interrogration of Orbitrap Data Files for Mass Spectrometer System Management \textit{J. Prot. Res.}. PMID: 30462513\\
	2018&Kovalchik, K.A., Moggridge, S., Chen, D.D.Y., Morin, G.B., \textbf{Hughes, C.S.\textsuperscript{$\diamond$}} Parsing and Quantification of Raw Orbitrap Mass Spectrometer Data using RawQuant \textit{J. Prot. Res.}. PMID: 29682972\\
	2018&Moggridge, S., Sorensen, P.H., Morin, G.B., \textbf{Hughes, C.S.\textsuperscript{$\diamond$}} Extending the Compatibility of the SP3 Paramagnetic Processing Approach for Proteomics \textit{J. Prot. Res.}. PMID: 29565595\\
	2018&\textbf{Hughes, C.S.}, Morin, G. Using Public Data for Comparative Proteome Analysis in Precision Medicine Studies \textit{Proteomics}. PMID: 28887829\\
	2017&\textbf{Hughes, C.S.}, Spicer, V., Krokhin, O.V., Morin, G.B., Investigating Acquisition Performance on the Orbitrap Fusion When Using Tandem MS/MS/MS Scanning with Isobaric Tags \textit{J. Prot. Res.}. PMID: 28418257\\
	2017&\textbf{Hughes, C.S.}, Zhu, C., Spicer, V., Krokhin, O.V., Morin, G. Evaluating the Characteristics of Reporter Ion Signal Acquired
	in the Orbitrap Analyzer for Isobaric Mass Tag Proteome Quantification Experiments \textit{J. Prot. Res.}. PMID: 28418254\\
	2017&Tien, J.F., Mazloomian, A., Cheng, S.G., \textbf{Hughes, C.S.} \textit{et al}. CDK12 regulates alternative last exon mRNA splicing and promotes breast cancer cell invasion \textit{Nucleic Acids Res.}. PMID:28334900\\
	2017&Funnell, T., Tasaki, S., Oloumi, A., Araki, S., Kong, E., Yap, D., Nakayama, Y., \textbf{Hughes, C.S.} \textit{et al}. CLK-dependent exon
	recognition and conjoined gene formation revealed with a novel small molecule inhibitor \textit{Nat. Commun.}.PMID: 28232751\\
	2016&\textbf{Hughes, C.S.}, McConechy, M., Cochrane, D., Nazeran, T., Karnezis, A., Huntsman, D., Morin, G. Biomarker Discovery from High Resolution Proteomic Analysis of Fixed Ovarian Tumor Tissue Samples. \textit{Scientific Reports}. PMID: 27713570\\
	2016&Virant-Klun, I., Leicht, S., \textbf{Hughes, C.S.}, Krijgsveld, J. Identification of maturation-specific proteins by single-cell proteomics of human oocytes. \textit{Mol. Cell. Proteomics} PMID: 27215607\\

\end{tabular}

\newpage

\noindent
\begin{tabular}{L!{\VRule}R}
	2016&Gupta, I., Villanyi, Z., Kassem, S., \textbf{Hughes, C.S.}, Panasenko, O.O., Steinmetz, L.M., Collart, M.A. Translational Capacity of a Cell is Determined during Transcription Elongation via the Ccr4-Not Complex. \textit{Cell Reports} PMID: 27184853\\
	2014&\textbf{Hughes, C.S.}, Foehr, S., Garfield, D., Furlong, E.E., Steinmetz, L.M., Krijgsveld, J. Ultrasensitive proteome analysis using paramagnetic bead
	technology. \textit{Molecular Systems Biology}. PMID: 25358341\\
	2014&Radan, L., \textbf{Hughes, C.S.} \textit{et al}. Microenvironmental regulation of telomerase isoforms in human embryonic stem cells. \textit{Stem Cells Dev.}. PMID: 24749509\\
	2012&\textbf{Hughes,C.S.} and Krijgsveld, J. Developments in quantitative mass spectrometry for the analysis of proteome dynamics. \textit{Trends in Biotechnology}. PMID: 23107010\\
	%2012&Peng, Y., Bocker, M.T., Holm, J., Toh, W.S., \textbf{Hughes, C.S.} \textit{et al}. Human fibroblast matrices bio-assembled under macromolecular crowding support stable propagation of human embryonic stem cells. \textit{J. Tissue Eng. Regen. Med}. PMID: 22761168\\
	2012&\textbf{Hughes, C.S.} \textit{et al}. Mass spectrometry-based proteomic analysis of the matrix microenvironment in pluripotent stem cell culture.
	\textit{Mol. Cell. Prot.}. PMID: 23023296\\
	2011&\textbf{Hughes, C.S.} \textit{et al}. Proteomic analysis of extracellular matrices used in stem cell culture. \textit{Proteomics}. PMID: 21834137\\
	%2011&\textbf{Hughes, C.S.}* and Nuhn, A.* \textit{et al}. Proteomics of Human Embryonic Stem Cells. \textit{Proteomics}. PMID: 21225999\\
	2010&\textbf{Hughes, C.S.} \textit{et al}. Matrigel: A complex protein mixture required for optimal growth of cell culture. \textit{Proteomics}.
	PMID: 20162561\\
	%2010&\textbf{Hughes, C.S.} \textit{et al}. De novo sequencing methods in proteomics. \textit{Methods in Mol. Biol.}. PMID: 200013367\\[1ex]
	2009&Bendall, S.C., \textbf{Hughes, C.S.} \textit{et al}. An enhanced mass spectrometry approach reveals human embryonic stem cell growth factors in
	culture. \textit{Mol.Cell. Prot.}. PMID: 18936058\\
	2008&Bendall, S.C.*, \textbf{Hughes, C.S.}* \textit{et al}. Prevention of amino acid conversion in SILAC experiments with embryonic stem cells. \textit{Mol. Cell. Prot.}. PMID: 18487603\\
	%2007&Wu, J., \textbf{Hughes, C.S.} \textit{et al}. High-throughput cytochrome
	%P450 inhibition assays using laser diode thermal desorption-atmospheric
	%pressure chemical ionization-tandem mass spectrometry. \textit{Anal. Chem.}.
	%PMID: 17497828\\[1ex]
	&* - denotes co-first authorship.\\[1ex]
	&$\diamond$ - denotes senior authorship.\\[1ex]
\end{tabular}


\section*{Patents}
\begin{tabular}{L!{\VRule}R}
	2014& \textbf{Title:} Proteomic sample preparation using paramagnetic
	beads\\
	&\textbf{Inventors:} \textbf{Hughes, C.S.}*, Krijgsveld, J.,
	Steinmetz, L.\\
	&\textbf{Publication number:} WO2015118152A1, US20170074869A1, CA2938907A1, EP3102612A1\\
	&\textbf{Filing date:} 2015-02-09\\
	&* - denotes majority inventor
\end{tabular}

\section*{Selected Grants and Awards}
\begin{tabular}{L!{\VRule}R}
	2019& \textbf{Title:} Small equipment acquisition - Biocomp sucrose gradient profiler fluorescence reader, \textbf{Agency:} BCCF, \textbf{Role:} co-applicant\\
	2018& \textbf{Title:} Small equipment acquisition - Agilent HPLC, \textbf{Agency:} BCCF, \textbf{Role:} co-applicant\\
	2016& \textbf{Title:} Fixed tissue proteomics (FTP) applied to create a pragmatic clinical decision aid for endometrial cancer, \textbf{Agency:} CCSRI, \textbf{Role:} co-applicant\\
	2016& \textbf{Title:} Breast cancer classification and marker identification by comprehensive proteomic analysis, \textbf{Agency:} CBCF, \textbf{Role:} co-applicant\\
	2016& \textbf{Title:} High-resolution analysis of phenotypic fitness using genome-wide CRISPR editing coupled to quantitative mass spectrometry, \textbf{Agency:} BCPN, \textbf{Role:} co-applicant\\
	2016& \textbf{Title:} Small equipment acquisition - Bioruptor Pico sonicator, \textbf{Agency:} BCCF, \textbf{Role:} co-applicant\\
	2016& \textbf{Title:} Small equipment acquisition - Sutter laser capillary puller, \textbf{Agency:} BCCF, \textbf{Role:} co-applicant\\
	2014&European Interdisciplinary Post-Doctoral Fellowship\\
	%2009&NSERC Canadian Graduate Scholarship - Doctoral\\
	%2008&NSERC Canadian Graduate Scholarship - Masters\\
\end{tabular}

\section*{Selected Presentations}
\begin{tabular}{L!{\VRule}R}
	Poster&Canadian Cancer Research Conference 2019 \textbf{Title:} A subcellular atlas of mRNA translation machinery reveals a novel role for the RNA-binding protein YB-1\\
	Oral&Personalized Oncogenomics Clinical Research Session 2016 \textbf{Title:} Proteomics and Metabolomics in Personalized Oncogenomics\\
	Oral&BCPN 2016 \textbf{Title:} High Resolution Proteomic Analysis of Ovarian FFPE Tumour Tissues using TMT-MS3 on an Orbitrap Fusion for Clinical Research\\
	Poster&ASMS 2015 \textbf{Title:} Enhanced processing of FFPE tissue for clinical proteomics using SP3\\
	Poster&ASMS 2014 \textbf{Title:} Single-tube sample preparation workflows for Ultra-sensitive Proteomics\\
	Oral&Nordic Proteomics Meeting 2014 \textbf{Title:} Single-tube sample preparation workflows for Ultra-sensitive Proteomics\\
	Poster&Proteostasis Discussion 2013 \textbf{Title:} Studying the dynamics of proteome homeostasis using hyperplexed mass spectrometry\\
	%Poster&HUPO 2011 \textbf{Title:} Analysis of the Human Embryonic Stem Cell Depositome\\
	Poster&ISSCR 2011 \textbf{Title:} Proteomic Analysis of the Human Embryonic Stem Cell Depositome\\
	Poster&HUPO 2010 \textbf{Title:} Matrigel: A complex mixture required for optimal growth of cell culture\\
	%Poster&HUPO 2010 \textbf{Title:} Analysis and quantitation of a glycogen synthase kinase-3 knockout embryonic stem cell line to a depth of 3500 proteins\\
	%Poster&ASMS 2009 \textbf{Title:} Analysis and quantitation of a glycogen synthase kinase-3 knockout embryonic stem cell line to a depth of 3500 proteins\\
	%Oral&ASMS 2008 \textbf{Title:} Prevention of amino acid conversion in SILAC experiments with embryonic stem cells\\
\end{tabular}

\newpage
\section*{Post-Doctoral Experience}
\begin{tabular}{L!{\VRule}R}
	2012--2014&{\bf EIPOD Post-Doctoral Researcher}\\
	&{\bf European Molecular Biology Laboratory}\\
	&I worked in the groups of Drs. Jeroen Krijgsveld and Lars Steinmetz towards a goal of performing integrative studies that utilize genomics, transcriptomics, and proteomics to study dynamic molecular systems.\\
\end{tabular}\\

\begin{addmargin}[7.5em]{2em}% 1em left, 2em right
	From these projects we achieved:
	\begin{itemize}
		\item Invention of an enhanced platform for unbiased and universal proteomics
		sample handling in conventional and ultra-sensitive applications, leading to the
		first reported screen of single-\textit{Drosophila} embryos across
		developmental stages.
		\item In-depth insight into the multi-layered regulation of cellular phenotype
		making use of protein turnover data acquired across a population of
		\textgreater200 individual yeast segregants.
	\end{itemize}
\end{addmargin}

\section*{Professional Experience}
\begin{tabular}{L!{\VRule}R}
	2014--2018&{\bf Group Leader and Proteomics Platform Manager}\\
	&{British Columbia Genome Sciences Centre}\\
	&Development, execution, analysis, and management of research projects utilizing proteomics in a diverse array of clinical experimental models, from large-scale patient cohort profiling to precision medicine clinical cancer trials. Administrative and techinical management of the proteomics core facility, including training, employee management, and funding acquisition.\\
	2006--2006&{\bf Research Technician}\\
	&{Merck Frosst Canada}\\
	&Worked on validating a new mass
	spectrometer ion source based on thermal desorption.\\
	2006--2006&{\bf Research Technician}\\
	&{Radient Technologies Inc.}\\
	&Worked on projects related to utilizing large-scale
	strategies to purify active compounds from natural products.\\
	2004--2005&{\bf Research Technician}\\
	&{MDS Sciex}\\
	&Worked on projects related to profiling carcinogenic food dyes and excipient analysis of analgesic drugs.\\
\end{tabular}

\section*{Selected Education and Professional Development}
\begin{tabular}{L!{\VRule}R}
	2007--2012&{\bf PhD in Biochemistry, Western University, Canada}\\
	&Thesis title: Proteomics of Human Stem Cell Derived Matrices\\
	2002--2007&Honors Bachelor of Science in Biology, University of Waterloo, Canada\\
	2013--2013&Next-generation Sequencing Workshop, European Bioinformatics Institute, UK\\
	2013--2013&Introduction to Python, EMBL, Germany\\
	2013--2013&Introduction to MATLAB and image analysis, EMBL, Germany\\
	%2010--2010&Stem Cell Culture Training Course, Stem Cell Technologies, Canada\\
	%2009--2009&Quantitative Proteomics Workshop, American Society for Mass
	%Spectrometry\\
	%2008--2008&FT-ICR Mass Spectrometry Workshop, American Society for Mass
	%Spectrometry\\
\end{tabular}

\section*{References}
Additional references available upon request.\\
\begin{addmargin}[3em]{2em}% 1em left, 2em right

	\noindent\textbf{Dr. Poul Sorensen}\\
	\textit{British Columbia Cancer Agency, Vancouver, Canada}\\
	Role: Senior Scientist\\
	Email: psorensen@bccrc.ca\\
	Phone: +1 604-675-8202\\
	Relationship: Current supervisor\\

	\noindent\textbf{Dr. Gregg Morin}\\
	\textit{British Columbia Cancer Agency, Vancouver, Canada}\\
	Role: Head of Proteomics\\
	Email: gmorin@bcgsc.ca\\
	Phone: +1 604-675-8154\\
	Relationship: Previous supervisor\\

	\noindent\textbf{Dr. Jeroen Krijgsveld}\\
	\textit{German Cancer Research Centre, Heidelberg, Germany}\\
	Role: Group leader, Proteomics of Stem Cells and Cancer\\
	Email: j.krijgsveld@dkfz-heidelberg.de\\
	Phone: +49 06221 421720\\
	Relationship: Post-doctoral supervisor\\

	%\noindent\textbf{Dr. Lars Steinmetz}\\
	%\textit{European Molecular Biology Laboratory, Heidelberg, Germany}\\
	%Role: Group leader\\
	%Email: lars.steinmetz@embl.de\\
	%Phone: +49 06221 387-8389\\
	%Preferred contact method: Email\\
	%Relationship: Post-Doctoral Supervisor\\

	\noindent\textbf{Dr. Gilles Lajoie}\\
	\textit{Western University, London, Canada}\\
	Role: Group leader and head of Proteomics Core Facility\\
	Email: galajoie@gmail.com\\
	Phone: +1 519-661-3054 ext.83054\\
	Relationship: Ph.D. supervisor
\end{addmargin}
\end{document}
