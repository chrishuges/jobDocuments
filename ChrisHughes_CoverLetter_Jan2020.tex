\documentclass[11pt]{article}
\usepackage{array, xcolor, lipsum, bibentry, parallel}
\usepackage[margin=3cm]{geometry}
\usepackage{setspace}
\usepackage{tabularx}
\usepackage{paralist}
\usepackage{scrextend}
\usepackage[colorlinks = true, linkcolor = blue, urlcolor  = blue, citecolor = blue, anchorcolor = blue]{hyperref}


\title{\bfseries\Huge Christopher Hughes}
\author{}
\date{}

\definecolor{lightgray}{gray}{0.8}
\newcolumntype{L}{>{\raggedleft}p{0.14\textwidth}}
\newcolumntype{R}{p{0.8\textwidth}}
\newcommand\VRule{\color{lightgray}\vrule width 0.5pt}


\begin{document}

	\newpage
	\setcounter{page}{1}
	\noindent

\begin{LARGE}

	\noindent\textbf{Christopher Hughes}\vspace{1pt}

\end{LARGE}


\noindent\rule{\textwidth}{1pt}\vspace{9pt}

\noindent
\begin{minipage}{0.5\textwidth}
	\begin{flushleft}
		chughes@bcgsc.ca\\
		Nationality: Canadian\\ \href{https://scholar.google.com/citations?user=jPSwBGwAAAAJ}{Google Scholar Citations}
	\end{flushleft}
\end{minipage}%
\begin{minipage}{0.5\textwidth}
	\begin{flushright}
		2477 Carolina Street, Unit 303\\
		V5T 0G8 Vancouver, Canada\\ \href{https://www.linkedin.com/in/christopher-hughes-612460133/}{LinkedIn Profile}
	\end{flushright}
\end{minipage}\vspace{18pt}

\noindent
Personal and research statement:\vspace{12pt}

I am a trained biochemist with extensive experience working in the life and clinical sciences fields using a wide array of molecular biology, sequencing, proteomics, metabolomics, and bioinformatics technologies. I strive to acquire a position that will enable me to apply my knowledge and expertise towards facilitating impactful studies that influence our knowledge of public health and complex diseases, such as cancer. In addition, I am increasingly driven to understand and implement ways to better translate research and scientific knowledge to be easily communicated and understood by both my colleagues and the general public. It is based on these interests that am applying to the XXX position at CHU Saint Justine, because as I have described below, I believe I have an optimal set of experience and skills to drive the rapid development and success of the facility while also continuing my growth as a scientist.\\

\noindent
During my graduate and post-graduate studies, the primary focus of my training has been centred in mass spectrometry and proteomics. I obtained my PhD at the University of Western Ontario working with Dr. Gilles Lajoie where I studied the interactions of human embryonic and induced pluripotent stem cells with their extracellular matrix environment using a combination of proteomic and molecular biology approaches. Upon completion of my PhD, I began a prestigious Interdisciplinary EIPOD fellowship at the European Molecular Biology Laborator (EMBL) in Heidelberg, Germany where I studied the dynamics of gene expression control in steady-state and stess conditions in the labs of Drs. Jeroen Krijgsveld and Lars Steinmetz using a combination of transcriptomic and proteomics approaches. Fortunately, both of these endeavours and my co-operative education placements during my undergraduate career have been largely interdisciplinary, allowing me to gain extensive experience working in the broad fields of genomics, transcriptomics, and metabolomics studying a diverse range of model systems.\\

\noindent
 After my time at EMBL, I pursued an exciting opportunity as the Proteomics Platform Manager in the Genome Sciences Centre (GSC) within the British Columbia Cancer Agency. The Proteomics Platform at the GSC operates primarily as a core facility providing a wide range of mass spectrometry and related wet- and dry-lab services for a diverse collection of basic and clinical researchers both within the institute and across Canada. My primary role this position included involvement in conception, technical development, funding acquisition, project management, data analysis, and the overall provision of high-quality service and collaborative work that caters to a diverse crowd of researchers. When I joined the GSC, the Proteomics Platform was largely unestablished and serviced only a few individuals within the GSC who were directly familiar with the technology. Within a year of my joining, I had set up and optimized the purchased instrumentation to achieve robust and reliable sample processing or data acquisition, developed a range of validated pipelines for sample and data processing, laid out extensive cost models and logging tools for recapture of funds, built collaborative relationships with other researchers and industry, published peer-reviewed literature based on the clinical applications of our work, and acquired grant funding for instrumentation and research. As a result of these efforts and outreach performed in these early stages, we were able to substantially expand the user base of the Proteomics Platform to an extent where we had, and continue to have, a committed sample queue of over 2-years. In the following years, we continued to publish research and technical development studies and acquire grant funding for research and instrumentation, allowing for the hiring of new members and most recently, the acquisition of \$2 million dollars from the BC Cancer Foundation for new instrumentation based on the need generated via our achievements. The core now operates as an efficient self-sustaining operation\\

\noindent
 After success in this manager position and in pursuit of a new challenge, I transitioned to a staff scientist and mass spectrometry service specialist position in the group of Dr. Poul Sorensen at the British Columbia Cancer Agency. In this senior role I am able to conceive, pursue funding for, and execute projects that currently relate to the study of different aspects of transcriptional and translational control of cellular phenotype in bone and childhood cancers. Aside from allowing me to apply my wet- and dry-lab skills to research projects relevant to my interests, this role has given me valuable experience relating to project and student management and grant acquisition from a supervisory viewpoint. Based on my previous work and my time in the group of Dr. Poul Sorensen, my current and future research interests are centered on the study of translational and metabolic control in Ewing sarcoma and the development of technologies that can improve the types and overall quality of data that can be obtained with a variety of technologies. I also have a keen interest in how to apply what we have learned from the extensive study of diseases like cancer to other areas, such as healty aging.\\

\noindent
I have attached my curriculum vitae for your consideration following this letter.
%\noindent
%Best regards,\\
%\LARGE{\textit{Christopher Hughes}}

\end{document}
